\section{Introduction}
In many situations, visual representations of matrices facilitate the
understanding of linear algebra properties, relations, and operations
enormously.  This package provides simple tools to bring such representations
to \LaTeX.  For instance,
\[
  \drawmatrix[upper]A \;
  \drawmatrix[width=.5]X +
  \drawmatrix[width=.5]X \;
  \drawmatrix[upper, size=.5, bbox height=1]B =
  \drawmatrix[width=.5]C
\]
is typeset as follows:
\begin{dispListing*}{listing only}
\[
  \drawmatrix[upper]A \;
  \drawmatrix[width=.5]X +
  \drawmatrix[width=.5]X \;
  \drawmatrix[upper, size=.5, bbox height=1]B =
  \drawmatrix[width=.5]C
\]
\end{dispListing*}


\section{Drawing Matrices}
\begin{docCommand}{drawmatrix}{\oarg{options}\marg{label}} 
    Draws a matrix labeled \meta{label}.  The optional \meta{options}, which
    modify various aspects of drawn matrix through PGF's key-value system, are
    introduced in the following sections.
\end{docCommand}

The label is typeset in the surrounding mode and style.
\begin{dispExample*}{righthand width=.35\textwidth}
    $\drawmatrix A$
    {\bfseries\drawmatrix A}
    {\large\drawmatrix A}
\end{dispExample*}
In equations, parentheses (spanned with \cs{left} and \cs{right}), subscripts,
and superscripts naturally extend to the drawn shape: 
\begin{dispExample*}{righthand width=.35\textwidth}
    $\left(
    \drawmatrix A_i + 
    \drawmatrix B^{-1}
    \right) 
    \drawmatrix C$  
\end{dispExample*}
Used in matrix products, a little space (\cs;) helps to yield a more natural
result: 
\begin{dispExample}
    $\drawmatrix A \drawmatrix B$ 

    $\drawmatrix A \; \drawmatrix B$
\end{dispExample}

\begin{docDmKey}{label text}{=\meta{text}}{no default, initially \meta{label}}
    Stores the label text.  It overwrites \refCom{drawmatrix}'s \meta{label}
    argument.
    \begin{dispExample}
        $\drawmatrix[label text=B]A$
    \end{dispExample}
\end{docDmKey}


\subsection{Size}

\begin{docDmKey}[][docmulti]{height}{=\meta{dimension}}{no default, initially 1}
\end{docDmKey}
\begin{docDmKey}{width}{=\meta{dimension}}{no default, initially 1}
    Width and height of the drawmatrix in \TikZ's coordinate system |canvas|.
    May be given in units such as |em| or |cm|.
    \begin{dispExample}
        \drawmatrix[width=.5]A
        \drawmatrix[width=2ex]B
        \drawmatrix[height=.35cm]C
    \end{dispExample}
\end{docDmKey}

A width or height of |0| are useful to represent vectors:
\begin{dispExample}
    \drawmatrix[width=0]x
\end{dispExample}

\begin{docDmKey}{size}{=\meta{dimension}}{style, no default}
    Shortcut for setting both \refKey{/drawmatrix/height} and
    \refKey{/drawmatrix/width} to the same \meta{dimension}, resulting in a
    square matrix.
\end{docDmKey}


\subsection{Shape}

By default matrices are rectangular.


\subsubsection{Triangular and Trapezoidal Matrices}
\begin{docDmKey}[][docmulti]{lower}{}{style, no value, initially unset}
\end{docDmKey}
\begin{docDmKey}{upper}{}{style, no value, initially unset}
    Result in, respectively, lower- and upper-triangular matrices.  Non-square
    matrices become trapezoidal.
    \begin{dispExample}
        \drawmatrix[lower]L
        \drawmatrix[upper, width=1.5]U
    \end{dispExample}
\end{docDmKey}


\subsubsection{Banded Matrices}

\begin{docDmKey}[][docmulti]{lower banded}{}{style, no value, initially unset}
\end{docDmKey}
\begin{docDmKey}{upper banded}{}{style, no value}
    Draw matrices as banded with bandwidth |0.3|.
    \begin{dispExample}
        \drawmatrix[lower banded]B
        \drawmatrix[upper banded]B
    \end{dispExample}
\end{docDmKey}

\begin{docDmKey}{banded}{}{style, no value}
    Shortcut for setting both \refKey{/drawmatrix/lower banded} and
    \refKey{/drawmatrix/upper banded}.
    \begin{dispExample}
        \drawmatrix[banded]B
    \end{dispExample}
\end{docDmKey}

\begin{docDmKey}[][docmulti]{lower bandwidth}{=\meta{dimension}}{no default, initially empty}
\end{docDmKey}
\begin{docDmKey}{upper bandwidth}{=\meta{dimension}}{no default, initially empty}
    The bandwidths, i.e., the horizontal/vertical extent from the diagonal.
    \begin{dispExample}
        \drawmatrix[lower bandwidth=.5]B
        \drawmatrix[upper bandwidth=.5]B
    \end{dispExample}
\end{docDmKey}
\begin{docDmKey}{bandwidth}{=\meta{dimension}}{style, no default}
    Shortcut for setting both \refKey{/drawmatrix/lower bandwidth} and
    \refKey{/drawmatrix/upper bandwidth}.
    \begin{dispExample}
        \drawmatrix[bandwidth=.5]B
    \end{dispExample}
\end{docDmKey}

Banding on rectangular matrices applies to the smaller of the two dimensions:
\begin{dispExample}
    \drawmatrix[banded, width=.8]B
    \drawmatrix[upper banded, height=.7]B
\end{dispExample}

\refKey{/drawmatrix/banded} can be combined with \refKey{/drawmatrix/lower} or
\refKey{/drawmatrix/upper} to draw the intersection of both shapes.
\begin{dispExample}
    \drawmatrix[banded, lower]L
\end{dispExample}


\subsubsection{Diagonal Matrices}

\begin{docDmKey}{diag}{}{style, no value}
    Shortcut for \refKey{/drawmatrix/bandwidth}|=0|.
    \begin{dispExample}
        \drawmatrix[diag]D
    \end{dispExample}
\end{docDmKey}


\subsubsection{Super- and Subscripts}

\begin{docDmKey}{label base}{=\meta{text}}{no default, initially empty}
    Defines the label to be centered in the drawmatrix, and to which the actual
    \refKey{/drawmatrix/label text} is aligned.  This feature is useful to,
    e.g., draw centered labels with exponents:
    \begin{dispExample}
        $\drawmatrix[size=.5]{A^T}$
        $\drawmatrix[size=.5, label base=A]{A^T}$
    \end{dispExample}
\end{docDmKey}

\begin{docDmKey}{label base anchor}{=\meta{anchor}}{no default, initially |base west|}
    The anchor of the \refKey{/drawmatrix/label text} with respect to the
    \refKey{/drawmatrix/label base}.
    \begin{dispExample}
        $\drawmatrix[size=.5, label base=A, label base anchor=base east]{^0_1A}$
    \end{dispExample}
\end{docDmKey}

\begin{docDmKey}{exponent}{=\meta{text}}{style, no default}
    Shortcut to add an exponent to matrix without offsetting the label.  It sets
    the \refKey{/drawmatrix/label base} to the current \refKey{/drawmatrix/label
    text} and adds the exponent \meta{text} to \refKey{/drawmatrix/label text}.
    \begin{dispExample}
       $\drawmatrix[size=.5, exponent=T]A$
    \end{dispExample}
\end{docDmKey}


\subsection{Colors and Style}

By default, matrices are drawn in gray and filled white.  The \TikZ keys
|draw=|\meta{color} and |fill=|\meta{color} change these colors.  In fact, all
keys not recognized by this package are passed to the \TikZ \cs{filldraw} command
drawing the matrix.
\begin{dispExample}
   \drawmatrix[fill=yellow, draw=blue]A
   \drawmatrix[very thick, dashed]A
\end{dispExample}


\subsection{The Bounding Box}

All matrices are contained in a rectangular bounding box.  

\begin{docDmKey}{bbox}{}{style, no default, initially empty}
    Add options to the \TikZ \cs{node} that is the bounding box..
    This is useful to, e.g., to visualize the 0 entries in the matrix:
    \begin{dispExample}
        \drawmatrix[lower, bbox/.append style={fill=blue!10}]L
    \end{dispExample}
\end{docDmKey}
\begin{docDmKey}{bbox style}{=\marg{style}}{style, no default}
    Shortcut for \refKey{/drawmatrix/bbox}|/.append style=|\meta{style}.
\end{docDmKey}


\begin{docDmKey}[][docmulti]{bbox height}{=\meta{dimension}}{no default, initially empty}
\end{docDmKey}
\begin{docDmKey}{bbox width}{=\meta{dimension}}{no default, initially empty}
    Explicitly set the height and width of the bounding box.
    If unset, the bounding box is just large enough to contain the matrix.
    The label of the matrix (and thus the alignment with respect to the
    surrounding text) are by default fixed at the center\footnote{See
    \refKey{/drawmatrix/label anchor}.} of the bounding box, while the matrix is
    positioned at its top-left corner.
    \begin{dispExample}
        \drawmatrixset{bbox style={fill=blue!10}}

        \drawmatrix[bbox width=2, bbox height=1.5]A
    \end{dispExample}
\end{docDmKey}

\begin{docDmKey}{bbox size}{=\meta{dimension}}{style, no default}
    Shortcut for setting \refKey{/drawmatrix/bbox height} and
    \refKey{/drawmatrix/bbox width} to the same value.
    \begin{dispExample}
        \drawmatrixset{bbox style={fill=blue!10}}

        \drawmatrix[bbox size=1.5]A
    \end{dispExample}
\end{docDmKey}

\begin{docDmKey}[][docmulti]{offset height}{=\meta{dimension}}{no default, initially |0|}
\end{docDmKey}
\begin{docDmKey}{offset width}{=\meta{dimension}}{no default, initially |0|}
    Sets the vertical and horizontal offset of the drawn matrix within its
    bounding box.
    \begin{dispExample}
       \drawmatrixset{bbox style={fill=blue!10}}

       \drawmatrix[bbox size=2, offset width=.5, offset height=.75]A
    \end{dispExample}
\end{docDmKey}

\begin{docDmKey}{offset}{=\meta{dimension}}{style, no default}
    Shortcut for setting \refKey{/drawmatrix/offset height} and
    \refKey{/drawmatrix/bbox width} to the same value.
    \begin{dispExample}
       \drawmatrixset{bbox style={fill=blue!10}}

       \drawmatrix[bbox size=2, offset=.5]A
    \end{dispExample}
\end{docDmKey}


\subsection{Coordinate System Transformations}

\begin{docDmKey}{scale}{=\meta{factor}}{style, no default}
    Scales all dimensions passed to a matrix.  Can be used repeatedly to
    multiply scales
    \begin{dispExample}
        \drawmatrix[scale=.6]A 
        \drawmatrix[scale=.6, width=.5]B
        \drawmatrix[scale=.7, scale=.7]B
    \end{dispExample}
\end{docDmKey}

\begin{docDmKey}[][docmulti]x{=\meta{value}}{style, no default}
\end{docDmKey}
\begin{docDmKey}y{=\meta{value}}{style, no default}
    Define the coordinate system for all unit-less dimensions.
    \begin{dispExample}
        \drawmatrix[x=.6cm, y=.4cm]A
        \drawmatrix[x=.6cm, y=.4cm, width=1cm]B
    \end{dispExample}
\end{docDmKey}


\subsection{Position of the Label and Baseline}

By default, the label's |mid| is positioned at the bounding box's |center| and
its |base| is used as the whole drawing's baseline.  

\begin{docDmKey}{label anchor}{=\meta{anchor}}{style, no default, initially |mid|}
    Set the anchor of label's \TikZ \cs{node}.
    \begin{dispExample}
        \drawmatrix[label anchor=north]A
    \end{dispExample}
\end{docDmKey}

\begin{docDmKey}{label pos}{=\meta{position}}{style, no default, initially |bbox.center|}
    Define the position of the label's \TikZ \cs{node} within the picture.  The
    following nodes and their anchors are available: |bbox| (the bounding box)
    and the |matrix| (matrix itself).
    \begin{dispExample}
        \drawmatrix[label pos=bbox.south]A
        \drawmatrix[label pos=matrix.north west]B
    \end{dispExample}
\end{docDmKey}

\begin{docDmKey}{baseline}{=\meta{position}}{style, no default, initially |label.base|}
    Specify how the picture is vertically aligned with the surrounding text's
    baseline.  Options are the same anchors as for \refKey{/drawmatrix/label
    pos} and anchors of |label| (the label).
    \begin{dispExample}
        A
        \drawmatrix[baseline=label.north]A
        \drawmatrix[baseline=bbox.south]A
    \end{dispExample}
\end{docDmKey}


\section{Changing Defaults}

\begin{docCommand}{drawmatrixset}{\marg{options}}
    Applies options to to all following uses of \refCom{drawmatrix} within the current
    scope.
    \begin{dispExample}
       \drawmatrixset{height=.5, lower}

       $\drawmatrix A \; \drawmatrix B$
    \end{dispExample}
\end{docCommand}

\begin{docDmKey}[][docmulti]{every picture}{}{style, no value}
\end{docDmKey}
\begin{docDmKey}[][docmulti]{every bbox}{}{style, no value}
\end{docDmKey}
\begin{docDmKey}[][docmulti]{every drawmatrix}{}{style, no value}
\end{docDmKey}
\begin{docDmKey}[][docmulti]{every label}{}{style, no value}
\end{docDmKey}
\begin{docDmKey}{every label}{}{style, no value}
    Settings for all drawmatrix pictures, bounding boxes, matrices, and labels.
    Options should be added not with |/.style=| but with |/.append style=| to
    avoid messing with internals.
    \begin{dispExample}
        \drawmatrixset{every drawmatrix/.append style={rounded corners=5pt}}

        $\drawmatrix A \; \drawmatrix[lower]B$
    \end{dispExample}
\end{docDmKey}


\section{Externalization}

\refCom{drawmatrix} behaves as any other \TikZ picture, therefore when
externalization is enabled, all matrix visualizations are also externalized.
However, since there are usually many \refCom{drawmatrix} pictures, each of
which is very small and fast to produce, their externalization would mean a
tremendous overhead.

\begin{docDmKey}{externalize}{=true\textbar false}{no default, initially |true|}
    Explicitly disables externalization for all \refCom{drawmatrix} pictures. It
    does not enable externalization.
\end{docDmKey}
