% \iffalse meta-comment
%
% Copyright (C) 2016 by Elmar Peise
% -----------------------------------------
%
% This package is distributed under the MIT license.
%
% \fi
%
% \iffalse
%<package>\NeedsTeXFormat{LaTeX2e}[1999/12/01]
%<package>\ProvidesPackage{drawmatrix}[2016/08/25 v1.2.0 drawmatrix package]
%
%<*driver>
\documentclass{ltxdoc}
\usepackage{hypdoc}
\usepackage{drawmatrix}
\usepackage{xspace}
\EnableCrossrefs
\RecordChanges
\CodelineIndex
\OnlyDescription
\begin{document}
  \DocInput{drawmatrix.dtx}
\end{document}
%</driver>
% \fi
%
% \CheckSum{131}
%
% \CharacterTable
%   {Upper-case    \A\B\C\D\E\F\G\H\I\J\K\L\M\N\O\P\Q\R\S\T\U\V\W\X\Y\Z
%    Lower-case    \a\b\c\d\e\f\g\h\i\j\k\l\m\n\o\p\q\r\s\t\u\v\w\x\y\z
%    Digits        \0\1\2\3\4\5\6\7\8\9
%    Exclamation   \!     Double quote  \"     Hash (number) \#
%    Dollar        \$     Percent       \%     Ampersand     \&
%    Acute accent  \'     Left paren    \(     Right paren   \)
%    Asterisk      \*     Plus          \+     Comma         \,
%    Minus         \-     Point         \.     Solidus       \/
%    Colon         \:     Semicolon     \;     Less than     \<
%    Equals        \=     Greater than  \>     Question mark \?
%    Commercial at \@     Left bracket  \[     Backslash     \\
%    Right bracket \]     Circumflex    \^     Underscore    \_
%    Grave accent  \`     Left brace    \{     Vertical bar  \|
%    Right brace   \}     Tilde         \~}
%
% \changes{v1.0.0}{2014/02/23}{Initial Version}
% \changes{v1.0.1}{2014/09/08}{Bugfix: Collapsible bbox (label placement for
% vectors)}
% \changes{v1.0.2}{2015/03/20}{Bugfix: Bbox had a linewidth/2 offset}
% \changes{v1.0.3}{2015/11/18}{Introduced |diag| option}
% \changes{v1.1.0}{2015/11/26}{Separate lower and upper banding, externalization
% control, fixed matrix style collection, lots of simplifications, moved
% ``every'' keys to |/drawmatrix| path}
% \changes{v1.1.1}{2016/01/29}{Bugfix: Remove extra space after vectors}
% \changes{v1.2.0}{2016/08/25}{Add coordinate transformations}
%
%
% \GetFileInfo{drawmatrix.sty}
%
% \title{The \textsf{drawmatrix} package}
% \author{Elmar Peise \\ \texttt{peise@aices.rwth-aachen.de}}
% \date{\filedate \ \fileversion}
%
% \newenvironment{example}{
%   \begin{center}\begin{minipage}{.9\textwidth}
%       \scriptsize
% }{
%   \end{minipage}\end{center}
% }
% \newcommand\TikZ{Ti{\it k}Z\xspace}
%
% \maketitle
%
% \begin{abstract}
%   \textsf{drawmatrix} provides macros to visually represent matrices.  Various
%   options allow to change the visualizations, e.g., drawing rectangular,
%   triangular, or banded matrices.
% \end{abstract}
%
% \tableofcontents
%
% \section{Introduction}
% In many situations, visual representations of matrices facilitate the
% understanding of linear algebra properties, relations, and operations
% enormously.  This package provides simple tools to bring such representations
% to \LaTeX.  For instance,
% $$
%   \drawmatrix[upper]A \;
%   \drawmatrix[width=.5]X +
%   \drawmatrix[width=.5]X \;
%   \drawmatrix[upper, size=.5, bbox height=1]B =
%   \drawmatrix[width=.5]C
% $$
% is typeset as follows:
% \begin{verbatim}
% $$
%   \drawmatrix[upper]A \;
%   \drawmatrix[width=.5]X +
%   \drawmatrix[width=.5]X \;
%   \drawmatrix[upper, size=.5, bbox height=1]B =
%   \drawmatrix[width=.5]C
% $$
% \end{verbatim}
%
%
% \section{Drawing Matrices}
% \DescribeMacro\drawmatrix |\drawmatrix|\oarg{options}\marg{label} draws a
% matrix labeled \meta{label}: |\drawmatrix A| produces \drawmatrix A.  The
% \meta{options}, which modify various aspects of drawn matrix through PGF's
% key-value system, are introduced in the following sections.
%
% By default, the matrix is centered around its label, which is aligned with the
% surrounding text.  The label is typeset in the surrounding mode and style.
% \begin{example}
%   |$\drawmatrix A$|:
%    $\drawmatrix A$
%
%   |{\bf \drawmatrix A}|:
%    {\bf \drawmatrix A}
%
%   |{\large \drawmatrix A}|:
%    {\large \drawmatrix A}
% \end{example}
% In equations, parentheses (spanned with |\left| and |\right|), subscripts, and
% superscripts naturally extend to the drawn shape: $\left(\drawmatrix{A}_i +
% \drawmatrix{B}^{-1}\right) \drawmatrix{C}$.  Used in matrix products such as
% $\drawmatrix{A} \drawmatrix{B}$, a little space (|\;|) helps to yield a more
% natural result: $\drawmatrix{A} \; \drawmatrix{B}$.
%
% \subsection{Size}
% By default, matrices are $1 \times 1$ large in terms of \TikZ units.  The
% width and height of a matrix are set through, respectively,
% \DescribeMacro{height} |width=|\meta{dimension} and \DescribeMacro{width}
% |height=|\meta{dimension}.  A width or height of 0 are useful to represent
% vectors:
% \begin{example}
%   |\drawmatrix[width=0]A|:
%    \drawmatrix[width=0]A
% \end{example}
%
% \DescribeMacro{size} |size=|\meta{dimension} sets both the width and height to
% \meta{dimension}, resulting in a square matrix.
%
% \subsection{Shape}
% By default matrices are rectangular.
% \subsubsection{Triangular and Trapezoidal Matrices}
% Lower and upper triangular matrices are obtained by, respectively, setting the
% keys \DescribeMacro{lower} |lower| and \DescribeMacro{upper} |upper|.  Hereby,
% non-square matrices become trapezoidal.
% \begin{example}
%   |\drawmatrix[lower]L|:
%    \drawmatrix[lower]L
%
%   |\drawmatrix[upper, width=1.5]U|:
%    \drawmatrix[upper, width=1.5]U
% \end{example}
%
% \subsubsection{Banded Matrices}
% Matrices are drawn as banded through the key \DescribeMacro{banded} |banded|.
% The band width, i.e., the horizontal/vertical extent from the diagonal, is
% specified through \DescribeMacro{bandwidth} |bandwidth=|\meta{dimension}
% (default: 0.3);
% \begin{example}
%   |\drawmatrix[banded]B|:
%    \drawmatrix[banded]B
%
%   |\drawmatrix[bandwidth=.5]B|:
%    \drawmatrix[bandwidth=.5]B
% \end{example}
%
% Banding for the lower and upper part of the matrices can be specified
% separately through \DescribeMacro{lower banded} |lower banded| and
% \DescribeMacro{upper banded} |upper banded|.  Separate bandwidths are set
% through \DescribeMacro{lower bandwidth} |lower bandwidth=|\meta{dimension} and
% \DescribeMacro{upper bandwidth} |upper bandwidth=|\meta{dimension}:
% \begin{example}
%   |\drawmatrix[lower banded]B|:
%    \drawmatrix[lower banded]B
%
%   |\drawmatrix[lower bandwidth=.5, upper bandwidth=.2]B|:
%    \drawmatrix[lower bandwidth=.5, upper bandwidth=.2]B
% \end{example}
%
% Banding on rectangular matrices applies to the smaller of the two dimensions:
% \begin{example}
%   |\drawmatrix[banded, width=.8]B|:
%    \drawmatrix[banded, width=.8]B
%
%   |\drawmatrix[upper banded, height=.7]B|:
%    \drawmatrix[upper banded, height=.7]B
% \end{example}
%
% |banded| can be combined with |lower| or |upper| to draw the intersection of
% both shapes.
% \begin{example}
%   |\drawmatrix[banded, lower]L|:
%    \drawmatrix[banded, lower]L
% \end{example}
%
% \subsubsection{Diagonal Matrices}
% \DescribeMacro{diag} |diag| is a shorthand for |banded| with |bandwidth=0|:
% \begin{example}
%   |\drawmatrix[diag]D|:
%    \drawmatrix[diag]D
% \end{example}
%
% \subsection{Colors and Style}
% By default, matrices are drawn in gray and filled white.  The \TikZ keys
% |draw=|\meta{color} and |fill=|\meta{color} change these colors.  In fact, all
% keys not recognized by this package are passed to the \TikZ |\filldraw|
% command drawing the matrix.
% \begin{example}
%   |\drawmatrix[fill=yellow, draw=blue]A|:
%    \drawmatrix[fill=yellow, draw=blue]A
%
%   |\drawmatrix[very thick, dashed]A|:
%    \drawmatrix[very thick, dashed]A
% \end{example}
%
% \subsection{The Bounding Box}
% All matrices are contained in a rectangular bounding box.  To draw this
% bounding box (e.g., to visualize the 0 entries in the matrix), use
% \DescribeMacro{bbox style} |bbox style=|\marg{style};  this style is applied
% to the \TikZ |\node| that is the bounding box.
% \begin{example}
%   |\drawmatrix[lower, bbox style={fill=blue!10}]L|:
%    \drawmatrix[lower, bbox style={fill=blue!10}]L
% \end{example}
%
% By default, the bounding box is just large enough to contain the matrix. Its
% size is changed through the keys \DescribeMacro{bbox height}
% |bbox height=|\meta{dimension} and \DescribeMacro{bbox width}
% |bbox width=|\meta{dimension} (or \DescribeMacro{bbox size}
% |bbox size=|\meta{dimension} to set them both).  The label of the matrix (and
% thus the alignment with respect to the surrounding text) are fixed at the
% center of the bounding box, while the matrix is positioned at its top-left
% corner.
% \begin{example}
%   |\drawmatrixset{bbox style={fill=blue!10}}|
%    \drawmatrixset{bbox style={fill=blue!10}}
%
%   |\drawmatrix[bbox width=2, bbox height=1.5]A|:
%    \drawmatrix[bbox width=2, bbox height=1.5]A
% \end{example}
%
% \sloppy The matrix can be positioned within its bounding box through
% \DescribeMacro{offset height} |offset height=|\meta{dimension} and
% \DescribeMacro{offset width} |offset width=|\meta{dimension} (or just
% \DescribeMacro{offset} |offset=|\meta{dimension} to shift along the diagonal).
% \begin{example}
%   |\drawmatrixset{bbox style={fill=blue!10}}|
%    \drawmatrixset{bbox style={fill=blue!10}}
%
%   |\drawmatrix[bbox size=2, offset width=.5, offset height=.75]A|:
%    \drawmatrix[bbox size=2, offset width=.5, offset height=.75]A
% \end{example}
%
% \subsection{Coordinate system transformations}
%
% |scale=|\meta{factor} scales all dimensions passed to a matrix:
% \begin{example}
%  |\drawmatrix[scale=.6]A \drawmatrix[scale=.6, width=.5]B|:
%   \drawmatrix[scale=.6]A \drawmatrix[scale=.6, width=.5]B
% \end{example}
%
% |x=|\meta{value} and |y=|\meta{value} define the coordinate system for all
% unit-less dimensions.
% \begin{example}
%  |\drawmatrix[x=.6cm, y=.4cm]A \drawmatrix[x=.6cm, y=.4cm, width=1cm]B|:
%   \drawmatrix[x=.6cm, y=.4cm]A \drawmatrix[x=.6cm, y=.4cm, width=1cm]B
% \end{example}
%
%
% \subsection{Position of the Label and Baseline}
% By default, the label's |mid| is positioned at the bounding box's |center| and
% its |base| is used as the whole drawing's baseline.  This is controlled by the
% keys \DescribeMacro{label anchor} |label anchor=|\meta{anchor},
% \DescribeMacro{label pos} |label pos=|\meta{position}, and
% \DescribeMacro{baseline} |baseline=|\meta{position}. Here, \meta{position} has
% to be an anchor of one of the following nodes: |bbox| (the bounding box),
% |matrix| (the matrix itself), or |label| (the label).
% \begin{example}
%   |\drawmatrixset{bbox height=1, height=.5, bbox style={fill=blue!10}}|
%    \drawmatrixset{bbox height=1, height=.5, bbox style={fill=blue!10}}
%
%   |\drawmatrix[label pos=bbox.south, label anchor=south]A|:
%    \drawmatrix[label pos=bbox.south, label anchor=south]A
%
%   |\drawmatrix[label pos=matrix.north west]A|:
%    \drawmatrix[label pos=matrix.north west]A
%
%   |\drawmatrix[baseline=label.north]A|:
%    \drawmatrix[baseline=label.north]A
%
%   |\drawmatrix[baseline=bbox.south]A|:
%    \drawmatrix[baseline=bbox.south]A
% \end{example}
%
% \section{Changing Defaults}
% Specifying \meta{options} with \DescribeMacro{\drawmatrixset}
% |\drawmatrixset|\marg{options} applies them to all following uses of
% |\drawmatrix| within the current scope.
% \begin{example}
%   |\drawmatrixset{height=.5, lower}|
%    \drawmatrixset{height=.5, lower}
%
%   |$\drawmatrix A \; \drawmatrix B$|:
%    $\drawmatrix A \; \drawmatrix B$
% \end{example}
%
% Furthermore, \TikZ keys for the entire picture, the bounding box, the matrix
% itself and the label can be set through the styles \DescribeMacro{every bbox}
% |every bbox|, \DescribeMacro{every drawmatrix} |every drawmatrix|, and
% \DescribeMacro{every label} |every label|.
% \begin{example}
%    |\drawmatrixset{every drawmatrix/.append style={rounded corners=5pt}}|
%     \drawmatrixset{every drawmatrix/.append style={rounded corners=5pt}}
%
%    |$\drawmatrix A \; \drawmatrix[lower]B$|:
%     $\drawmatrix A \; \drawmatrix[lower]B$
% \end{example}
%
% \section{Externalization}
% |\drawmatrix| behaves as any other \TikZ picture, therefore when
% externalization is enabled, all matrix visualizations are also externalized.
% However, since there are usually many |\drawmatrix| pictures, each of which is
% very small and fast to produce, their externalization would mean a tremendous
% overhead.  To avoid this overhead without explicitly dis- and re-enabling
% externalization throughout the document, \DescribeMacro{externalize}
% |externalize=false| disables externalization for all |\drawmatrix| pictures:
% \begin{example}
%   |\drawmatrixset{externalize=false}|
%    \drawmatrixset{externalize=false}
% \end{example}
%
%
% \section{Implementation}
%
% \DoNotIndex{
%   \dm@bboxheight, \dm@bboxwidth, \dm@currname, \dm@currval, \dm@height,
%   \dm@labeltext, \dm@lowerbandwidth, \dm@minsize, \dm@offsetheight,
%   \dm@offsetwidth, \dm@upperbandwidth, \dm@width, \dm@zero, \RequirePackage,
%   \begin, \def, \edef, \else, \end, \expandafter, \fi, \filldraw,
%   \ifdrawmatrix@externalize, \ifmmode, \ifx, \let, \newcommand, \newif, \node,
%   \path, \pgfkeys, \pgfkeysalso, \pgfkeyscurrentname, \pgfkeyscurrentvalue,
%   \pgfkeysnovalue, \pgflinewidth, \pgfgetlastxy, \pgfmathsetlengthmacro,
%   \pgfresetboundingbox, \pgfqkeys, \tikz@library@external@loaded, \tikzset,
%   \undefined
% }
%
% This section describes the implementation details of the \textsf{drawmatrix}
% package.
%
% \subsection{Package: \TikZ}
%
% The \textsf{tikz} package is used for drawing.
%    \begin{macrocode}
\RequirePackage{tikz}
%    \end{macrocode}
%
% \subsection{If for externalization}
%
% \TeX{} |if| representing whether to explicitly disable \TikZ externalization.
% \begin{macro}{\ifdrawmatrix@externalize}
%    \begin{macrocode}
\newif\ifdrawmatrix@externalize
%    \end{macrocode}
% \end{macro}
%
% \subsection{Key Declarations and Defaults}
%
% We rely on PGF keys as much as we can.
%    \begin{macrocode}
\pgfkeys{
%    \end{macrocode}
%
% Everything happens in the path |/drawmatrix|.
%    \begin{macrocode}
    drawmatrix/.is family,
    drawmatrix/.cd,
%    \end{macrocode}
%
% \begin{macro}{picture}
% \begin{macro}{baseline}
%   |picture| is the style for the |\tikzpicture| in which the matrix is drawn.
%   |baseline| sets the baseline of the picture to a named coordinate of the
%   matrix (default: base of the label).
%    \begin{macrocode}
    picture/.style={},
    path/.style={},
    baseline/.style={picture/.append style={baseline=(drawmatrix #1)}},
    scale/.style={path/.append style={scale=#1}},
    x/.style={path/.append style={x=#1}},
    y/.style={path/.append style={y=#1}},
    baseline=label.base,
%    \end{macrocode}
% \end{macro}\end{macro}
%
% \begin{macro}{bbox}\begin{macro}{bbox style}
%   |bbox| is the style of the bounding box, to which |bbox style| appends keys.
%    \begin{macrocode}
    bbox/.style={},
    bbox style/.style={bbox/.append style={#1}},
%    \end{macrocode}
% \end{macro}\end{macro}
%
% \begin{macro}{bbox height}
% \begin{macro}{bbox width}
% \begin{macro}{bbox size}
%   |bbox height| and |bbox width| don't have default values.  |bbox size| sets
%   them both to the same value.
%    \begin{macrocode}
    bbox height/.initial,
    bbox width/.initial,
    bbox size/.style={bbox height=#1, bbox width=#1},
%    \end{macrocode}
% \end{macro}\end{macro}\end{macro}
%
% \begin{macro}{offset height}
% \begin{macro}{offset width}
% \begin{macro}{offset}
%   |offset height| and |offset width| are |0| by default. |offset| sets them
%   both to the same value.
%    \begin{macrocode}
    offset height/.initial=0,
    offset width/.initial=0,
    offset/.style={offset height=#1, offset width=#1},
%    \end{macrocode}
% \end{macro}\end{macro}\end{macro}
%
% \begin{macro}{height}
% \begin{macro}{width}
% \begin{macro}{size}
%   |width| and |height| are |1| (\TikZ unit) by default. |size| sets them both
%   to the same value.
%    \begin{macrocode}
    height/.initial=1,
    width/.initial=1,
    size/.style={height=#1, width=#1},
%    \end{macrocode}
% \end{macro}\end{macro}\end{macro}
%
% \begin{macro}{lower bandwidth}
% \begin{macro}{upper bandwidth}
% \begin{macro}{bandwidth}
%   The |lower bandwidth| and |upper bandwidth| don't have default values.
%   |bandwidth| sets them both to the same value.
%    \begin{macrocode}
    lower bandwidth/.initial,
    upper bandwidth/.initial,
    bandwidth/.style={lower bandwidth=#1, upper bandwidth=#1},
%    \end{macrocode}
% \end{macro}\end{macro}\end{macro}
%
% \begin{macro}{lower banded}
% \begin{macro}{upper banded}
% \begin{macro}{banded}
%   |lower banded| and |upper banded| are shortcuts to set the corresponding
%   bandwidths to the default value of |0.3| (\TikZ units). |banded| sets them
%   both.
%    \begin{macrocode}
    lower banded/.style={lower bandwidth=.3},
    upper banded/.style={upper bandwidth=.3},
    banded/.style={lower banded, upper banded},
%    \end{macrocode}
% \end{macro}\end{macro}\end{macro}
%
% \begin{macro}{lower}
% \begin{macro}{upper}
% \begin{macro}{diag}
%   |lower| and |upper| are implemented by setting the opposite bandwidth to
%   |0|. |diag| sets them both.
%    \begin{macrocode}
    lower/.style={upper bandwidth=0},
    upper/.style={lower bandwidth=0},
    diag/.style={lower, upper},
%    \end{macrocode}
% \end{macro}\end{macro}\end{macro}
%
% \begin{macro}{externalize}
%   |externalize| sets a \TeX{} if (default: |true| = behave as all pictures).
%    \begin{macrocode}
    externalize/.is if=drawmatrix@externalize,
    externalize=true,
%    \end{macrocode}
% \end{macro}
%
% \begin{macro}{label}
% \begin{macro}{label pos}
% \begin{macro}{label anchor}
%   |label| is the style for the label. |label pos| sets the label at a named
%   coordinate of the matrix (default: center of the bounding box).
%   |label anchor| sets the label's |anchor| (default: in the middle).
%    \begin{macrocode}
    label/.style={},
    label pos/.style={label/.append style={at=(drawmatrix #1)}},
    label pos=bbox.center,
    label anchor/.style={label/.append style={anchor=#1}},
    label anchor=mid,
%    \end{macrocode}
% \end{macro}\end{macro}\end{macro}
%
% Unknown keys are collected in |/drawmatrix/drawmatrix|.
%    \begin{macrocode}
    drawmatrix/.style={},
    .unknown/.code={%
        \let\dm@currname\pgfkeyscurrentname%
        \let\dm@currval\pgfkeyscurrentvalue%
        \ifx#1\pgfkeysnovalue\pgfkeysalso{
            drawmatrix/.append style/.expand once={\dm@currname}
        }\else\pgfkeysalso{
            drawmatrix/.append style/.expand twice={%
                \expandafter\dm@currname\expandafter=\dm@currval%
            }
        }\fi%
    },
%    \end{macrocode}
%
% \begin{macro}{every picture}
% \begin{macro}{every bbox}
% \begin{macro}{every drawmatrix}
% \begin{macro}{every label}
%   The default style for matrices: |every picture| applies to all
%   |\tikzpicture|s the matrices are drawn in, |every bbox| applies to all
%   bounding boxes, |every drawmatrix| applies to the matrices themselves, and
%   |every label| applies to the labels.
%    \begin{macrocode}
    every picture/.style={},
    every bbox/.style={
        name=drawmatrix bbox,
        inner sep=0,
    },
    every drawmatrix/.style={
        fill=white,
        draw=gray,
    },
    every label/.style={
        name=drawmatrix label,
        outer sep=0,
        inner sep=0,
    },
    every node/.style={
        name=drawmatrix matrix,
        outer sep=0,
        inner sep=0,
        anchor=north west,
    }
}
%    \end{macrocode}
% \end{macro}\end{macro}\end{macro}\end{macro}
%
% \subsection{User Macros}
%
% |\drawmatrixset| as a simple shortcut like |\tikzset|.
% \begin{macro}{\drawmatrixset}
%    \begin{macrocode}
\newcommand\drawmatrixset[1]{\pgfqkeys{/drawmatrix}{#1}}
%    \end{macrocode}
% \end{macro}
%
% Here we go, the main thing: |\drawmatrix|. First, apply the options and
% extract the sizes from the PGF keys.
% \begin{macro}{\drawmatrix}
%    \begin{macrocode}
\newcommand\drawmatrix[2][]{{%
    \drawmatrixset{
        #1,
        height/.get=\dm@height,
        width/.get=\dm@width,
        lower bandwidth/.get=\dm@lowerbandwidth,
        upper bandwidth/.get=\dm@upperbandwidth,
        offset height/.get=\dm@offsetheight,
        offset width/.get=\dm@offsetwidth,
        bbox height/.get=\dm@bboxheight,
        bbox width/.get=\dm@bboxwidth,
    }%
%    \end{macrocode}
%
% Prepare the label text (keep math mode).
%    \begin{macrocode}
    \ifmmode\def\dm@labeltext{$#2$}\else\def\dm@labeltext{#2}\fi%
%    \end{macrocode}
%
% Disable externalization if |externalize=false|.  Start the picture.
%    \begin{macrocode}
    \ifdrawmatrix@externalize\else%
        \ifx\tikz@library@external@loaded\undefined\else%
            \tikzset{external/export=false}%
        \fi%
    \fi%
    \begin{tikzpicture}[/drawmatrix/every picture, /drawmatrix/picture]
%    \end{macrocode}
%
% Parse width, height, the minimum dimension and zero for comparison purposes.
%    \begin{macrocode}
        \path[/drawmatrix/path] (\dm@width, \dm@height);
        \pgfgetlastxy\dm@width\dm@height
        \path[/drawmatrix/path] (\dm@offsetwidth, \dm@offsetheight);
        \pgfgetlastxy\dm@offsetwidth\dm@offsetheight
        \pgfmathsetlengthmacro\dm@minsize{min(\dm@width, \dm@height)}
        \pgfmathsetlengthmacro\dm@zero{0.0}
%    \end{macrocode}
%
% Prepare the band widths:  First, if the matrix is not banded, the bandwidth is
% set to the smaller matrix dimension.  Then, the band width is limited by this
% smaller dimension.
%    \begin{macrocode}
        \expandafter\ifx\dm@lowerbandwidth\pgfkeysnovalue
            \edef\dm@lowerbandwidth{\dm@minsize}
        \else
            \path[/drawmatrix/path] (\dm@lowerbandwidth, 0);
            \pgfgetlastxy\dm@lowerbandwidth\dm@zero
        \fi
        \expandafter\ifx\dm@upperbandwidth\pgfkeysnovalue
            \edef\dm@upperbandwidth{\dm@minsize}
        \else
            \path[/drawmatrix/path] (0, \dm@upperbandwidth);
            \pgfgetlastxy\dm@zero\dm@upperbandwidth
        \fi
        \pgfmathsetlengthmacro\dm@lowerbandwidth{
            min(\dm@minsize, \dm@lowerbandwidth)
        }
        \pgfmathsetlengthmacro\dm@upperbandwidth{
            min(\dm@minsize, \dm@upperbandwidth)
        }
%    \end{macrocode}
%
% Set the default bounding box size.
%    \begin{macrocode}
        \expandafter\ifx\dm@bboxheight\pgfkeysnovalue
            \pgfmathsetlengthmacro\dm@bboxheight{
                \dm@height + \dm@offsetheight
            }
        \else
            \path[/drawmatrix/path] (0, \dm@bboxheight);
            \pgfgetlastxy\dm@zero\dm@bboxheight
        \fi
        \expandafter\ifx\dm@bboxwidth\pgfkeysnovalue
            \pgfmathsetlengthmacro\dm@bboxwidth{
                \dm@width + \dm@offsetwidth
            }
        \else
            \path[/drawmatrix/path] (\dm@bboxwidth, 0);
            \pgfgetlastxy\dm@bboxwidth\dm@zero
        \fi
%    \end{macrocode}
%
% Reset the bounding box and begin with (drawing) the path for the bounding box.
%    \begin{macrocode}
        \pgfresetboundingbox
        \node[/drawmatrix/every bbox, /drawmatrix/bbox,
            minimum height=\dm@bboxheight,
            minimum width=\dm@bboxwidth] {};
%    \end{macrocode}
%
% Whether needed or not, declare all matrix corners.
%    \begin{macrocode}
        \path (drawmatrix bbox.north west)
            ++(\dm@offsetwidth, -\dm@offsetheight)
            ++(.5\pgflinewidth, -.5\pgflinewidth)
            coordinate (drawmatrix north west)
            ++(\dm@width, 0)
            +(-\dm@minsize + \dm@upperbandwidth, 0)
            coordinate (drawmatrix north)
            +(0, -\dm@minsize + \dm@upperbandwidth)
            coordinate (drawmatrix east)
            ++(0, -\dm@height)
            coordinate (drawmatrix south east)
            ++(-\dm@width, 0)
            +(\dm@minsize - \dm@lowerbandwidth, 0)
            coordinate (drawmatrix south)
            +(0, \dm@minsize - \dm@lowerbandwidth)
            coordinate (drawmatrix west);
%    \end{macrocode}
%
% Add an invisible node the size of the matrix.
%    \begin{macrocode}
        \node[/drawmatrix/every node,
            minimum height=\dm@height,
            minimum width=\dm@width]
            at (drawmatrix north west) {};
%    \end{macrocode}
%
% Now, draw only what is needed of the matrix.  Otherwise path modifications
% (e.g., such as rounded corners) might not work.
%    \begin{macrocode}
        \filldraw[/drawmatrix/every drawmatrix, /drawmatrix/drawmatrix]
            (drawmatrix north west)
            \ifx\dm@upperbandwidth\dm@zero
                \ifx\dm@width\dm@minsize\else -- (drawmatrix north) \fi
                \ifx\dm@height\dm@minsize\else -- (drawmatrix east) \fi
            \else
                -- (drawmatrix north)
                \ifx\dm@upperbandwidth\dm@minsize\else
                    -- (drawmatrix east)
                \fi
            \fi
            -- (drawmatrix south east)
            \ifx\dm@lowerbandwidth\dm@zero
                \ifx\dm@width\dm@minsize\else -- (drawmatrix south) \fi
                \ifx\dm@height\dm@minsize\else -- (drawmatrix west) \fi
            \else
                -- (drawmatrix south)
                \ifx\dm@lowerbandwidth\dm@minsize\else
                    -- (drawmatrix west)
                \fi
            \fi
            -- cycle;
%    \end{macrocode}
%
% The label.
%    \begin{macrocode}
        \node[/drawmatrix/every label, /drawmatrix/label]
            {\dm@labeltext};
    \end{tikzpicture}%
}}
%    \end{macrocode}
% \end{macro}
%
% \Finale
% \PrintIndex
% \PrintChanges
%
\endinput
