% \iffalse meta-comment
%
% Copyright (C) 2015 by Elmar Peise
% -----------------------------------------
%
% This file may be distributed and/or modified under the conditions of the
% LaTeX Project Public License, either version 1.2 of this license or (at your
% option) any later version.  The latest version of this license is an:
%
%   http://www.latex-project.org/lppl.txt
%
% and version 1.2 or later is part of all distributions of LaTeX version
% 1999/12/01 or later.
%
% \fi
%
% \iffalse
%<package>\NeedsTeXFormat{LaTeX2e}[1999/12/01]
%<package>\ProvidesPackage{drawmatrix}
%<package> [2015/11/18 v1.0.3 drawmatrix package]
%
%<*driver>
\documentclass{ltxdoc}
\usepackage{hypdoc}
\usepackage{drawmatrix}
\EnableCrossrefs
\RecordChanges
\CodelineIndex
\OnlyDescription
\begin{document}
  \DocInput{drawmatrix.dtx}
\end{document}
%</driver>
% \fi
%
% \CheckSum{131}
%
% \CharacterTable
%   {Upper-case    \A\B\C\D\E\F\G\H\I\J\K\L\M\N\O\P\Q\R\S\T\U\V\W\X\Y\Z
%    Lower-case    \a\b\c\d\e\f\g\h\i\j\k\l\m\n\o\p\q\r\s\t\u\v\w\x\y\z
%    Digits        \0\1\2\3\4\5\6\7\8\9
%    Exclamation   \!     Double quote  \"     Hash (number) \#
%    Dollar        \$     Percent       \%     Ampersand     \&
%    Acute accent  \'     Left paren    \(     Right paren   \)
%    Asterisk      \*     Plus          \+     Comma         \,
%    Minus         \-     Point         \.     Solidus       \/
%    Colon         \:     Semicolon     \;     Less than     \<
%    Equals        \=     Greater than  \>     Question mark \?
%    Commercial at \@     Left bracket  \[     Backslash     \\
%    Right bracket \]     Circumflex    \^     Underscore    \_
%    Grave accent  \`     Left brace    \{     Vertical bar  \|
%    Right brace   \}     Tilde         \~}
%
% \changes{v1.0.0}{2014/02/23}{Initial Version}
% \changes{v1.0.1}{2014/09/08}{Bugfix: Collapsible bbox (label placement for
% vectors)}
% \changes{v1.0.2}{2015/03/20}{Bugfix: Bbox had a linewidth/2 offset}
% \changes{v1.0.3}{2015/11/18}{Introduced |diag| option}
%
%
% \GetFileInfo{drawmatrix.sty}
%
% \title{The \textsf{drawmatrix} package}
% \author{Elmar Peise \\ \texttt{peise@aices.rwth-aachen.de}}
% \date{\filedate \ \fileversion}
%
% \newenvironment{example}{
%   \begin{center}
%   \begin{minipage}{.9\textwidth}
%       \scriptsize
% }{
%   \end{minipage}
%   \end{center}
% }
%
% \maketitle
%
% \begin{abstract}
%   This package provides macros to draw visual representations matrices.
% \end{abstract}
%
% \tableofcontents
%
% \section{Introduction}
% I think visually a lot.  So, when I work with matrices, I draw them --- on
% paper, on the blackboard, in documents, in presentations.  This package
% provides simple, yet powerful tools to do so in \LaTeX.  For instance,
% $$
%   \drawmatrix[upper]A \;
%   \drawmatrix[width=.5cm]X +
%   \drawmatrix[width=.5]X \;
%   \drawmatrix[upper, size=.5, bbox height=1]B =
%   \drawmatrix[width=.5]C
% $$
% is typeset with this package as follows:
% \begin{verbatim}
% $$
%   \drawmatrix[upper]A \;
%   \drawmatrix[width=.5]X +
%   \drawmatrix[width=.5]X \;
%   \drawmatrix[upper, size=.5, bbox height=1]B =
%   \drawmatrix[width=.5]C
% $$
% \end{verbatim}
%
%
% \section{Drawing Matrices}
% \DescribeMacro\drawmatrix |\drawmatrix|\oarg{options}\marg{label} draws a
% matrix labeled \meta{label}: |\drawmatrix{A}| produces \drawmatrix{A}.  The
% \meta{options}, which modify various aspects of drawn matrix through {\sc
% pgf}'s key-value system, are introduced in the following sections.
%
% The matrix is centered around its label, which is aligned with the surrounding
% text.  The label is typeset in the surrounding mode and style.
% \begin{example}
%   |$\drawmatrix A$|:
%    $\drawmatrix A$
%
%   |{\bf \drawmatrix A}|:
%    {\bf \drawmatrix A}
%
%   |{\large \drawmatrix A}|:
%    {\large \drawmatrix A}
% \end{example}
% In equations, parentheses (spanned with |\left| and |\right|), subscripts, and
% superscripts naturally extend to the drawn shape: $\left(\drawmatrix{A}_i +
% \drawmatrix{B}^{-1}\right) \drawmatrix{C}$.  Used in matrix products such as
% $\drawmatrix{A} \drawmatrix{B}$, a little space (|\;|) helps to yield a more
% natural result: $\drawmatrix{A} \; \drawmatrix{B}$.
%
% \subsection{Size}
% By default, matrices are 1cm$\times$1cm large.  The width and height of a
% matrix are set through, respectively, \DescribeMacro{height}
% |width=|\meta{expr} and \DescribeMacro{width} |height=|\meta{expr}.
% \meta{expr} is evaluated by {\sc pgf}'s mathematical engine; if it contains no
% units cm is implied, otherwise, all numbers without units are taken as in pt.
% \begin{example}
%   |\drawmatrix[height=.7, width=1cm + 10]A|:
%    \drawmatrix[height=.7, width=1cm + 10]A
% \end{example}
% A width or height of 0 result in vector representations:
% \begin{example}
%   |\drawmatrix[width=0]A|:
%    \drawmatrix[width=0]A
% \end{example}
%
% \DescribeMacro{size} |size=|\meta{expr} sets both the width and height of the
% drawn matrix to \meta{expr}, resulting in a square matrix.
%
% \subsection{Shape}
% By default matrices are rectangular.
% \subsubsection{Triangular and Trapezoidal}
% Lower and upper triangular (and trapezoidal) matrices are obtained by,
% respectively, setting the keys \DescribeMacro{lower} |lower| and
% \DescribeMacro{upper} |upper|.  Hereby, non-square matrices become
% trapezoidal.
% \begin{example}
%   |\drawmatrix[lower]L|:
%    \drawmatrix[lower]L
%
%   |\drawmatrix[upper, width=1.5]U|:
%    \drawmatrix[upper, width=1.5]U
% \end{example}
%
% \subsubsection{Banded matrices}
% Matrices are drawn as banded through the key \DescribeMacro{banded} |banded|.
% The width, i.e. the horizontal/vertical extent from the diagonal, of the band
% (default: 0.3cm) is specified through \DescribeMacro{bandwidth}
% |bandwidth=|\meta{expr};
% \begin{example}
%   |\drawmatrix[banded]B|:
%    \drawmatrix[banded]B
%
%   |\drawmatrix[banded, bandwidth=.5]B|:
%    \drawmatrix[banded, bandwidth=.5]B
% \end{example}
%
% Banding for the lower and upper part of the matrices can be specified
% separately through \DescribeMacro{lower banded} |lower banded| and
% \DescribeMacro{upper banded} |upper banded|.  Separate bandwidths are set
% through \DescribeMacro{lower bandwidth} |lower bandwidth=|\meta{expr} and
% \DescribeMacro{upper bandwidth} |upper bandwidth=|\meta{expr}:
% \begin{example}
%   |\drawmatrix[lower banded]B|:
%    \drawmatrix[lower banded]B
%
%   |\drawmatrix[banded, lower bandwidth=.5, upper bandwidth=.2]B|:
%    \drawmatrix[banded, lower bandwidth=.5, upper bandwidth=.2]B
% \end{example}
%
% Banding on rectangular matrices applies to the smaller of the two dimensions:
% \begin{example}
%   |\drawmatrix[banded, width=.8]B|:
%    \drawmatrix[banded, width=.8]B
%
%   |\drawmatrix[upper banded, height=.7]B|:
%    \drawmatrix[upper banded, height=.7]B
% \end{example}
%
% |banded| can be combined with |lower| or |upper| to draw the intersection of
% both shapes.
% \begin{example}
%   |\drawmatrix[banded, lower]L|:
%    \drawmatrix[banded, lower]L
% \end{example}
%
% {\em Implementation detail}: |lower| (and |upper|) are implemented as
% shorthands for |upper banded| with |upper bandwidth=0| (and |lower banded|
% width |lower bandwidth=0|).
%
% \subsection{Diagonal matrices}
% \DescribeMacro{diag} is a shorthand for |banded| with |bandwidth=0|:
% \begin{example}
%   |\drawmatrix[diag]D|:
%    \drawmatrix[diag]D
% \end{example}
%
% \subsection{Colors and Style}
% By default, matrices are drawn in gray and filled white.  The TikZ keys
% \DescribeMacro{draw} |draw=|\meta{color} and \DescribeMacro{fill}
% |fill=|\meta{color} are used to set these colors.  In fact, all keys not
% recognized by this package are passed to the TikZ |\filldraw| command drawing
% the matrix.
% \begin{example}
%   |\drawmatrix[fill=yellow, draw=blue]A|:
%    \drawmatrix[fill=yellow, draw=blue]A
%
%   |\drawmatrix[very thick, dashed]A|:
%    \drawmatrix[very thick, dashed]A
% \end{example}
%
% \subsection{The Bounding Box}
% All matrices are are contained in a rectangular bounding box.  To draw this
% bounding box (e.g. to visualize the 0 entries in the matrix),
% \DescribeMacro{bbox style} |bbox style=|\marg{style} is used;  this style
% is applied to the TikZ |\node| that is the bounding box.
% \begin{example}
%   |\drawmatrix[lower, bbox style={fill=blue!10}]L|:
%    \drawmatrix[lower, bbox style={fill=blue!10}]L
% \end{example}
%
% By default, the bounding box is just large enough to contain the matrix. Its
% size is changed through the keys
% \DescribeMacro{bbox height} |bbox height=|\meta{expr} and
% \DescribeMacro{bbox width} |bbox width=|\meta{expr} (or
% \DescribeMacro{bbox size} |bbox size=|\meta{expr} to set them both).  The
% label of the matrix (and thus the alignment with respect to the surrounding
% text) are fixed at the center of the bounding box, while the matrix is
% positioned at its top-left corner.
% \begin{example}
%   |\drawmatrixset{bbox style={fill=blue!10}}|
%    \drawmatrixset{bbox style={fill=blue!10}}
%
%   |\drawmatrix[bbox width=2, bbox height=1.5]A|:
%    \drawmatrix[bbox width=2, bbox height=1.5]A
% \end{example}
%
% \sloppy The matrix can be positioned within its bounding box through
% \DescribeMacro{offset height} |offset height=|\meta{expr} and
% \DescribeMacro{offset width} |offset width=|\meta{expr} (or just
% \DescribeMacro{offset} |offset=|\meta{expr} to shift along the diagonal).
% \begin{example}
%   |\drawmatrixset{bbox style={fill=blue!10}}|
%    \drawmatrixset{bbox style={fill=blue!10}}
%
%   |\drawmatrix[bbox size=2, offset width=.5, offset height=.75]A|:
%    \drawmatrix[bbox size=2, offset width=.5, offset height=.75]A
% \end{example}
%
% \subsection{Position of the Label and Baseline}
% By default, the label's |mid| is positioned at the bounding box's |center| and
% its |base| is used as the whole drawing's baseline.
% This is controlled by the keys
% \DescribeMacro{label anchor} |label anchor=|\meta{anchor},
% \DescribeMacro{label pos} |label pos=|\meta{position}, and
% \DescribeMacro{baseline} |baseline=|\meta{position}. Here, \meta{position} has
% to be an anchor of one of the following nodes: |bbox| (the bounding box),
% |matrix| (the matrix itself), or |label| (the label).
% \begin{example}
%   |\drawmatrixset{bbox height=1, height=.5, bbox style={fill=blue!10}}|
%    \drawmatrixset{bbox height=1, height=.5, bbox style={fill=blue!10}}
%
%   |\drawmatrix[label pos=bbox.south, label anchor=south]A|:
%    \drawmatrix[label pos=bbox.south, label anchor=south]A
%
%   |\drawmatrix[label pos=matrix.north west]A|:
%    \drawmatrix[label pos=matrix.north west]A
%
%   |\drawmatrix[baseline=label.north]A|:
%    \drawmatrix[baseline=label.north]A
%
%   |\drawmatrix[baseline=bbox.south]A|:
%    \drawmatrix[baseline=bbox.south]A
% \end{example}
%
% \section{Changing Defaults}
% Specifying \meta{options} with \DescribeMacro{\drawmatrixset}\marg{options}
% applies them to all following |\drawmatrix|'s within the current scope.
% \begin{example}
%   |\drawmatrixset{height=.5, lower}|
%    \drawmatrixset{height=.5, lower}
%
%   |$\drawmatrix A \; \drawmatrix B$|:
%    $\drawmatrix A \; \drawmatrix B$
% \end{example}
%
% Furthermore, TikZ keys for the entire picture, the bounding box, the matrix
% itself and the label can be set through the TikZ styles
% \DescribeMacro{every drawmatrix bbox} |every drawmatrix bbox|,
% \DescribeMacro{every drawmatrix} |every drawmatrix|, and
% \DescribeMacro{every drawmatrix label} |every drawmatrix label|.
% \begin{example}
%    |\tikzset{every drawmatrix/.append style={draw=blue}}|
%     \tikzset{every drawmatrix/.append style={draw=blue}}
%
%    |$\drawmatrix A \; \drawmatrix B$|:
%     $\drawmatrix A \; \drawmatrix B$
% \end{example}
%
%
%^^A \StopEventually{
%^^A   \PrintChanges
%^^A   \PrintIndex
%^^A }
%
% \section{Implementation}
%
% \DoNotIndex{\begin, \csname, \def, \edef, \else, \end, \endcsname,
%   \expandafter, \fi, \filldraw, \ifmmode, \ifpgfmathunitsdeclared, \ifx, \let,
%   \newcommand, \newif, \node, \path, \pgfkeys, \pgfkeyscurrentkeyRAW,
%   \pgfkeyscurrentvalue, \pgfkeysnovalue, \pgfmathparse, \pgfmathresult,
%   \RequirePackage, \tikzset, \undefined}
%
% This section describes the implementation details of the \textsf{drawmatrix}
% package, stepping through the code line by line.
%
% \subsection{Tikz and Styles}
%
% The \textsf{tikz} package is used for drawing.
%    \begin{macrocode}
\RequirePackage{tikz}
%    \end{macrocode}
%
% Declare the default style for matrices.
%    \begin{macrocode}
\tikzset{
    every drawmatrix picture/.style={
    },
    every drawmatrix bbox/.style={
        name=drawmatrix bbox,
        inner sep=0,
    },
    every drawmatrix/.style={
        fill=white,
        draw=gray,
    },
    every drawmatrix label/.style={
        name=drawmatrix label,
        outer sep=0,
        inner sep=0,
    }
}
%    \end{macrocode}
%
% \subsection{String Constants and Ifs}
%
%    \begin{macrocode}
\newif\ifdrawmatrix@upperbanded
\newif\ifdrawmatrix@lowerbanded
\newif\ifdrawmatrix@mmode
%    \end{macrocode}
%
% \subsection{Key Declarations and Defaults}
%
% Helper macro to extract sizes.
%
%    \begin{macrocode}
\def\drawmatrix@setlength[#1]#2{%
    \pgfmathparse{#2}%
    \ifpgfmathunitsdeclared%
        \expandafter\edef\csname drawmatrix@#1\endcsname%
        {\pgfmathresult pt}%
    \else%
        \expandafter\edef\csname drawmatrix@#1\endcsname%
        {\pgfmathresult cm}%
    \fi%
}
%    \end{macrocode}
%
% Set up the path |/drawmatrix|.
%
%    \begin{macrocode}
\pgfkeys{
    drawmatris/.is family,
    drawmatrix/.cd,
%    \end{macrocode}
%
% The picture.
%
%    \begin{macrocode}
    picture/.style={},
    baseline/.style={picture/.append style={baseline=(drawmatrix #1)}},
%    \end{macrocode}
%
% The bounding box.
%
%    \begin{macrocode}
    bbox/.style={},
    bbox style/.style={bbox/.append style={#1}},
    bbox height/.code=\drawmatrix@setlength[bboxheight]{#1},
    bbox width/.code=\drawmatrix@setlength[bboxwidth]{#1},
    bbox size/.style={bbox height=#1, bbox width=#1},
%    \end{macrocode}
%
% The offset from the bounding box.
%
%    \begin{macrocode}
    offset height/.code=\drawmatrix@setlength[offsetheight]{#1},
    offset width/.code=\drawmatrix@setlength[offsetwidth]{#1},
    offset/.style={offset height=#1, offset width=#1},
%    \end{macrocode}
%
% The matrix size.
%
%    \begin{macrocode}
    drawmatrix/.style={},
    height/.code=\drawmatrix@setlength[height]{#1},
    width/.code=\drawmatrix@setlength[width]{#1},
    size/.style={height=#1, width=#1},
%    \end{macrocode}
%
% The matrix shape.
%
%    \begin{macrocode}
    lower banded/.is if=drawmatrix@lowerbanded,
    upper banded/.is if=drawmatrix@upperbanded,
    banded/.style={lower banded=#1, upper banded=#1},
    lower bandwidth/.code=\drawmatrix@setlength[lowerbandwidth]{#1},
    upper bandwidth/.code=\drawmatrix@setlength[upperbandwidth]{#1},
    bandwidth/.style={lower bandwidth=#1, upper bandwidth=#1},
    lower/.style={upper banded, upper bandwidth=0},
    upper/.style={lower banded, lower bandwidth=0},
    diag/.style={lower, upper},
%    \end{macrocode}
%
% Collect unknown keys in |/drawmatrix/drawmatrix|.
%
%    \begin{macrocode}
    .unknown/.code={%
        \let\drawmatrix@currkey\pgfkeyscurrentkeyRAW%
        \let\drawmatrix@currval\pgfkeyscurrentvalue%
        \ifx#1\pgfkeysnovalue%
            \pgfkeys{
                /drawmatrix/drawmatrix/.append style/.expand once={
                    \drawmatrix@currkey
                }
            }%
        \else%
            \pgfkeys{
                /drawmatrix/drawmatrix/.append style/.expand twice={
                    \expandafter\drawmatrix@currkey\expandafter=%
                    \drawmatrix@currval
                }
            }%
        \fi%
    },
%    \end{macrocode}
%
% The label.
%
%    \begin{macrocode}
    label/.style={},
    label pos/.style={label/.append style={at=(drawmatrix #1)}},
    label anchor/.style={label/.append style={anchor=#1}},
%    \end{macrocode}
%
% Finally, set the default keys.
%
%    \begin{macrocode}
    baseline=label.base,
    size=1, offset=0,
    banded=false, bandwidth=.3,
    label pos=bbox.center, label anchor=mid,
}
%    \end{macrocode}
%
% \subsection{User Macros}
%
% |\drawmatrixset| as a simple shortcut like |\tikzset|.
%
%    \begin{macrocode}
\newcommand\drawmatrixset[1]{\pgfkeys{drawmatrix/.cd, #1}}
%    \end{macrocode}
%
% Here we go, the main thing: |\drawmatrix|. First, we check if we are in math
% mode and apply the options.
%
%    \begin{macrocode}
\newcommand\drawmatrix[2][]{{%
    \ifmmode\drawmatrix@mmodetrue\else\drawmatrix@mmodefalse\fi%
    \drawmatrixset{#1}%
%    \end{macrocode}
%
% Next, we prepare the band widths:  First, if the matrix is not banded, the
% bandwidth is set to the smaller matrix dimension.  Then, the band width is
% limited by this smaller dimension.
%
%    \begin{macrocode}
    \pgfmathparse{min(\drawmatrix@width, \drawmatrix@height)}%
    \edef\drawmatrix@minsize{\pgfmathresult pt}%
    \ifdrawmatrix@lowerbanded\else%
        \edef\drawmatrix@lowerbandwidth{\drawmatrix@minsize}%
    \fi%
    \pgfmathparse{min(\drawmatrix@minsize, \drawmatrix@lowerbandwidth)}%
    \edef\drawmatrix@lowerbandwidth{\pgfmathresult pt}%
    \ifdrawmatrix@upperbanded\else%
        \edef\drawmatrix@upperbandwidth{\drawmatrix@minsize}%
    \fi%
    \pgfmathparse{min(\drawmatrix@minsize, \drawmatrix@upperbandwidth)}%
    \edef\drawmatrix@upperbandwidth{\pgfmathresult pt}%
%    \end{macrocode}
%
% Next, we set the default bounding box size.
%
%    \begin{macrocode}
    \ifx\drawmatrix@bboxheight\undefined%
        \pgfmathparse{\drawmatrix@height + \drawmatrix@offsetheight}%
        \edef\drawmatrix@bboxheight{\pgfmathresult pt}%
    \fi%
    \ifx\drawmatrix@bboxwidth\undefined%
        \pgfmathparse{\drawmatrix@width + \drawmatrix@offsetwidth}%
        \edef\drawmatrix@bboxwidth{\pgfmathresult pt}%
    \fi%
%    \end{macrocode}
%
% We begin with (drawing) the path for the bounding box.
%
%    \begin{macrocode}
    \begin{tikzpicture}[every drawmatrix picture, /drawmatrix/picture]
        \node[every drawmatrix bbox, /drawmatrix/bbox,
            minimum height=\drawmatrix@bboxheight,
            minimum width=\drawmatrix@bboxwidth] {};
%    \end{macrocode}
%
% For the matrix itself, we first declare all its corners and then draw it.
%
%    \begin{macrocode}
        \path (drawmatrix bbox.north west)
            ++(\drawmatrix@offsetwidth + .5\pgflinewidth,
               -\drawmatrix@offsetheight - .5\pgflinewidth)
            coordinate (drawmatrix north west)
            +(\drawmatrix@width, 0)
            coordinate (drawmatrix north east)
            +(\drawmatrix@width, -\drawmatrix@height)
            coordinate (drawmatrix south east)
            +(0, -\drawmatrix@height)
            coordinate (drawmatrix south west)
            (drawmatrix north east)
            +(-\drawmatrix@minsize + \drawmatrix@upperbandwidth, 0)
            coordinate (drawmatrix north)
            +(0, -\drawmatrix@minsize + \drawmatrix@upperbandwidth)
            coordinate (drawmatrix east)
            (drawmatrix south west)
            +(\drawmatrix@minsize - \drawmatrix@lowerbandwidth, 0)
            coordinate (drawmatrix south)
            +(0, \drawmatrix@minsize - \drawmatrix@lowerbandwidth)
            coordinate (drawmatrix west)
            ;
        \filldraw[every drawmatrix, /drawmatrix/drawmatrix]
            (drawmatrix north west)
            -- (drawmatrix north)
            -- (drawmatrix east)
            -- (drawmatrix south east)
            -- (drawmatrix south)
            -- (drawmatrix west)
            -- cycle;
        \node[minimum height=\drawmatrix@height,
            minimum width=\drawmatrix@width] (drawmatrix matrix) {};
%    \end{macrocode}
%
% The label.
%
%    \begin{macrocode}
        \node[every drawmatrix label, /drawmatrix/label] {%
            \ifdrawmatrix@mmode$#2$\else#2\fi%
        };
    \end{tikzpicture}%
}}
%    \end{macrocode}
% \PrintIndex
%
% \Finale
\endinput
